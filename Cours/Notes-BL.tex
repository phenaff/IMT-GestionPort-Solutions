% Options for packages loaded elsewhere
\PassOptionsToPackage{unicode}{hyperref}
\PassOptionsToPackage{hyphens}{url}
%
\documentclass[
]{article}
\usepackage{lmodern}
\usepackage{amsmath}
\usepackage{ifxetex,ifluatex}
\ifnum 0\ifxetex 1\fi\ifluatex 1\fi=0 % if pdftex
  \usepackage[T1]{fontenc}
  \usepackage[utf8]{inputenc}
  \usepackage{textcomp} % provide euro and other symbols
  \usepackage{amssymb}
\else % if luatex or xetex
  \usepackage{unicode-math}
  \defaultfontfeatures{Scale=MatchLowercase}
  \defaultfontfeatures[\rmfamily]{Ligatures=TeX,Scale=1}
\fi
% Use upquote if available, for straight quotes in verbatim environments
\IfFileExists{upquote.sty}{\usepackage{upquote}}{}
\IfFileExists{microtype.sty}{% use microtype if available
  \usepackage[]{microtype}
  \UseMicrotypeSet[protrusion]{basicmath} % disable protrusion for tt fonts
}{}
\makeatletter
\@ifundefined{KOMAClassName}{% if non-KOMA class
  \IfFileExists{parskip.sty}{%
    \usepackage{parskip}
  }{% else
    \setlength{\parindent}{0pt}
    \setlength{\parskip}{6pt plus 2pt minus 1pt}}
}{% if KOMA class
  \KOMAoptions{parskip=half}}
\makeatother
\usepackage{xcolor}
\IfFileExists{xurl.sty}{\usepackage{xurl}}{} % add URL line breaks if available
\IfFileExists{bookmark.sty}{\usepackage{bookmark}}{\usepackage{hyperref}}
\hypersetup{
  pdftitle={Black-Litterman},
  hidelinks,
  pdfcreator={LaTeX via pandoc}}
\urlstyle{same} % disable monospaced font for URLs
\usepackage[margin=1in]{geometry}
\usepackage{color}
\usepackage{fancyvrb}
\newcommand{\VerbBar}{|}
\newcommand{\VERB}{\Verb[commandchars=\\\{\}]}
\DefineVerbatimEnvironment{Highlighting}{Verbatim}{commandchars=\\\{\}}
% Add ',fontsize=\small' for more characters per line
\usepackage{framed}
\definecolor{shadecolor}{RGB}{248,248,248}
\newenvironment{Shaded}{\begin{snugshade}}{\end{snugshade}}
\newcommand{\AlertTok}[1]{\textcolor[rgb]{0.94,0.16,0.16}{#1}}
\newcommand{\AnnotationTok}[1]{\textcolor[rgb]{0.56,0.35,0.01}{\textbf{\textit{#1}}}}
\newcommand{\AttributeTok}[1]{\textcolor[rgb]{0.77,0.63,0.00}{#1}}
\newcommand{\BaseNTok}[1]{\textcolor[rgb]{0.00,0.00,0.81}{#1}}
\newcommand{\BuiltInTok}[1]{#1}
\newcommand{\CharTok}[1]{\textcolor[rgb]{0.31,0.60,0.02}{#1}}
\newcommand{\CommentTok}[1]{\textcolor[rgb]{0.56,0.35,0.01}{\textit{#1}}}
\newcommand{\CommentVarTok}[1]{\textcolor[rgb]{0.56,0.35,0.01}{\textbf{\textit{#1}}}}
\newcommand{\ConstantTok}[1]{\textcolor[rgb]{0.00,0.00,0.00}{#1}}
\newcommand{\ControlFlowTok}[1]{\textcolor[rgb]{0.13,0.29,0.53}{\textbf{#1}}}
\newcommand{\DataTypeTok}[1]{\textcolor[rgb]{0.13,0.29,0.53}{#1}}
\newcommand{\DecValTok}[1]{\textcolor[rgb]{0.00,0.00,0.81}{#1}}
\newcommand{\DocumentationTok}[1]{\textcolor[rgb]{0.56,0.35,0.01}{\textbf{\textit{#1}}}}
\newcommand{\ErrorTok}[1]{\textcolor[rgb]{0.64,0.00,0.00}{\textbf{#1}}}
\newcommand{\ExtensionTok}[1]{#1}
\newcommand{\FloatTok}[1]{\textcolor[rgb]{0.00,0.00,0.81}{#1}}
\newcommand{\FunctionTok}[1]{\textcolor[rgb]{0.00,0.00,0.00}{#1}}
\newcommand{\ImportTok}[1]{#1}
\newcommand{\InformationTok}[1]{\textcolor[rgb]{0.56,0.35,0.01}{\textbf{\textit{#1}}}}
\newcommand{\KeywordTok}[1]{\textcolor[rgb]{0.13,0.29,0.53}{\textbf{#1}}}
\newcommand{\NormalTok}[1]{#1}
\newcommand{\OperatorTok}[1]{\textcolor[rgb]{0.81,0.36,0.00}{\textbf{#1}}}
\newcommand{\OtherTok}[1]{\textcolor[rgb]{0.56,0.35,0.01}{#1}}
\newcommand{\PreprocessorTok}[1]{\textcolor[rgb]{0.56,0.35,0.01}{\textit{#1}}}
\newcommand{\RegionMarkerTok}[1]{#1}
\newcommand{\SpecialCharTok}[1]{\textcolor[rgb]{0.00,0.00,0.00}{#1}}
\newcommand{\SpecialStringTok}[1]{\textcolor[rgb]{0.31,0.60,0.02}{#1}}
\newcommand{\StringTok}[1]{\textcolor[rgb]{0.31,0.60,0.02}{#1}}
\newcommand{\VariableTok}[1]{\textcolor[rgb]{0.00,0.00,0.00}{#1}}
\newcommand{\VerbatimStringTok}[1]{\textcolor[rgb]{0.31,0.60,0.02}{#1}}
\newcommand{\WarningTok}[1]{\textcolor[rgb]{0.56,0.35,0.01}{\textbf{\textit{#1}}}}
\usepackage{graphicx}
\makeatletter
\def\maxwidth{\ifdim\Gin@nat@width>\linewidth\linewidth\else\Gin@nat@width\fi}
\def\maxheight{\ifdim\Gin@nat@height>\textheight\textheight\else\Gin@nat@height\fi}
\makeatother
% Scale images if necessary, so that they will not overflow the page
% margins by default, and it is still possible to overwrite the defaults
% using explicit options in \includegraphics[width, height, ...]{}
\setkeys{Gin}{width=\maxwidth,height=\maxheight,keepaspectratio}
% Set default figure placement to htbp
\makeatletter
\def\fps@figure{htbp}
\makeatother
\setlength{\emergencystretch}{3em} % prevent overfull lines
\providecommand{\tightlist}{%
  \setlength{\itemsep}{0pt}\setlength{\parskip}{0pt}}
\setcounter{secnumdepth}{-\maxdimen} % remove section numbering
\usepackage[utf8]{inputenc}
\usepackage{booktabs}
\usepackage{longtable}
\usepackage{array}
\usepackage{multirow}
\usepackage{wrapfig}
\usepackage{float}
\usepackage{colortbl}
\usepackage{pdflscape}
\usepackage{tabu}
\usepackage{threeparttable}
\usepackage{threeparttablex}
\usepackage[normalem]{ulem}
\usepackage{makecell}
\usepackage{xcolor}
\ifluatex
  \usepackage{selnolig}  % disable illegal ligatures
\fi

\title{Black-Litterman}
\usepackage{etoolbox}
\makeatletter
\providecommand{\subtitle}[1]{% add subtitle to \maketitle
  \apptocmd{\@title}{\par {\large #1 \par}}{}{}
}
\makeatother
\subtitle{The intuition behind Black-Litterman model portfolios}
\author{}
\date{\vspace{-2.5em}Février-Mars 2020}

\begin{document}
\maketitle

\begin{Shaded}
\begin{Highlighting}[]
\FunctionTok{library}\NormalTok{(xts)}
\FunctionTok{library}\NormalTok{(hornpa)}
\FunctionTok{library}\NormalTok{(lubridate)}
\FunctionTok{library}\NormalTok{(xtable)}
\FunctionTok{library}\NormalTok{(PerformanceAnalytics)}
\FunctionTok{library}\NormalTok{(TTR)}
\FunctionTok{library}\NormalTok{(lubridate)}
\FunctionTok{library}\NormalTok{(roll)}
\FunctionTok{library}\NormalTok{(Hmisc)}
\FunctionTok{library}\NormalTok{(nFactors)}
\FunctionTok{library}\NormalTok{(kableExtra)}
\FunctionTok{library}\NormalTok{(broom)}
\FunctionTok{library}\NormalTok{(quadprog)}
\end{Highlighting}
\end{Shaded}

\hypertarget{principle}{%
\section{Principle}\label{principle}}

Bayesian approach:

\begin{itemize}
\tightlist
\item
  The expected returns are random variables
\item
  CAPM equilibrium distribution as prior
\item
  additional probabilistic views combined with prior to get posterior
  distribution of expected return.
\end{itemize}

Distribution of asset returns:

\[
r \sim \mathcal{N}(\mu, \Sigma)
\]

Assume quadratic utility function:

\[
U(w) = w^T \Pi - \frac{\delta}{2} w^T \Sigma w
\] Solve first order conditions for optimality to get

\[
\Pi = \delta \Sigma w_{eq}
\]

The expected return \(\mu\) is also a random variable. The bayesian
prior is such that \[
\mu = \Pi + \epsilon^{(e)}
\] with \[
\epsilon^{(e)} \sim \mathcal{N}(0, \tau \Sigma)
\] where \(\tau\) is a scalar.

Views are expressed as portfolios whose returns are independent random
normal variables.

\[
P \mu = Q + \epsilon^{(v)}
\] with \[
\epsilon^{(v)} \sim \mathcal{N}(0, \Omega)
\]

\hypertarget{posterior-distribution}{%
\section{Posterior distribution}\label{posterior-distribution}}

\hypertarget{gls-linear-model}{%
\subsection{GLS linear model}\label{gls-linear-model}}

Consider the linear model \[
Y = X \beta + E
\] with \(\textrm{Cov}(E|X) = \Omega\)

\[
\hat{\beta} = (X^T \Omega^{-1}X)^{-1} X^T \Omega^{-1} Y
\]

\hypertarget{theils-estimatiom-method-for-posterior-distribution}{%
\subsection{Theil's estimatiom method for posterior
distribution}\label{theils-estimatiom-method-for-posterior-distribution}}

Prior distribution for return \[
\Pi = I \mu + \epsilon^{(e)}
\] Additional information: \[
Q = P \mu + \epsilon^{(v)}
\] Combine two equations: \[
\begin{bmatrix}
\Pi \\ 
Q
\end{bmatrix}
 = 
\begin{bmatrix}
I \\ 
P
\end{bmatrix}
\mu + 
\begin{bmatrix}
\epsilon^{(e)} \\ 
\epsilon^{(v)}  
\end{bmatrix}
\] Apply GLS:

\def\O{
\begin{bmatrix}
\tau \Sigma & \\
& \Omega
\end{bmatrix}
}

\def\X{
\begin{bmatrix}
I \\ 
P
\end{bmatrix}
}

\[
\mu^* = \left( 
\begin{bmatrix}
I \\ 
P
\end{bmatrix}
^T 
\begin{bmatrix}
\tau \Sigma & \\
& \Omega
\end{bmatrix}
^{-1} 
\begin{bmatrix}
I \\ 
P
\end{bmatrix}
\right)^{-1}
\begin{bmatrix}
I \\ 
P
\end{bmatrix}
^T
\begin{bmatrix}
\tau \Sigma & \\
& \Omega
\end{bmatrix}
^{-1} \begin{bmatrix}
\Pi \\ 
Q
\end{bmatrix}
\]

After algebraic manipulations:

Posterior mean of expected returns:

\[
\mu^* = \left[ (\tau \Sigma)^{-1} + P^T\Omega^{-1} P \right]^{-1} \left[ (\tau \Sigma)^{-1} \Pi + P^T\Omega^{-1} Q \right]
\]

Posterior covariance of expected returns:

\[
M^{-1} = \left[ (\tau \Sigma)^{-1} + P^T\Omega^{-1} P \right]^{-1}
\] Consequence: the posterior distribution of returns is \[
r \sim \mathcal{N}(\mu^*, \Sigma^*)
\] with \(\Sigma^* = \Sigma + M^{-1}\).

\hypertarget{portfolio-optimization}{%
\section{Portfolio optimization}\label{portfolio-optimization}}

One can now find the optimal weights by solving the classical
mean-variance problem:

\[
    \mbox{max} \   w^T\mu^*  - \frac{\delta}{2} w^T \Sigma^* w  \\
\]

the solution being: \[
w^* = \frac{1}{\delta} \Sigma^{*-1} \mu^*
\] See paper by He and Litterman for various manipulations of this last
equation.

\hypertarget{calculation}{%
\subsection{Calculation}\label{calculation}}

Code freely adapted from
\url{https://github.com/systematicinvestor/SIT}, but using the notation
of the paper.

Market data from He \& Litterman:

\begin{Shaded}
\begin{Highlighting}[]
\NormalTok{ data }\OtherTok{=}
\StringTok{\textquotesingle{}1,0.4880,0.4780,0.5150,0.4390,0.5120,0.4910}
\StringTok{ 0.4880,1,0.6640,0.6550,0.3100,0.6080,0.7790}
\StringTok{ 0.4780,0.6640,1,0.8610,0.3550,0.7830,0.6680}
\StringTok{ 0.5150,0.6550,0.8610,1,0.3540,0.7770,0.6530}
\StringTok{ 0.4390,0.3100,0.3550,0.3540,1,0.4050,0.3060}
\StringTok{ 0.5120,0.6080,0.7830,0.7770,0.4050,1,0.6520}
\StringTok{ 0.4910,0.7790,0.6680,0.6530,0.3060,0.6520,1\textquotesingle{}}
  
\NormalTok{  Corrmat }\OtherTok{=} \FunctionTok{matrix}\NormalTok{( }\FunctionTok{as.double}\NormalTok{(}\FunctionTok{spl}\NormalTok{( }\FunctionTok{gsub}\NormalTok{(}\StringTok{\textquotesingle{}}\SpecialCharTok{\textbackslash{}n}\StringTok{\textquotesingle{}}\NormalTok{, }\StringTok{\textquotesingle{},\textquotesingle{}}\NormalTok{, data), }\StringTok{\textquotesingle{},\textquotesingle{}}\NormalTok{)), }
                    \AttributeTok{nrow =} \FunctionTok{length}\NormalTok{(}\FunctionTok{spl}\NormalTok{(data, }\StringTok{\textquotesingle{}}\SpecialCharTok{\textbackslash{}n}\StringTok{\textquotesingle{}}\NormalTok{)), }\AttributeTok{byrow=}\ConstantTok{TRUE}\NormalTok{)}
  
\NormalTok{  stdevs }\OtherTok{=} \FunctionTok{c}\NormalTok{(}\FloatTok{16.0}\NormalTok{, }\FloatTok{20.3}\NormalTok{, }\FloatTok{24.8}\NormalTok{, }\FloatTok{27.1}\NormalTok{, }\FloatTok{21.0}\NormalTok{,  }\FloatTok{20.0}\NormalTok{, }\FloatTok{18.7}\NormalTok{)}\SpecialCharTok{/}\DecValTok{100}
\NormalTok{  w.eq }\OtherTok{=} \FunctionTok{c}\NormalTok{(}\FloatTok{1.6}\NormalTok{, }\FloatTok{2.2}\NormalTok{, }\FloatTok{5.2}\NormalTok{, }\FloatTok{5.5}\NormalTok{, }\FloatTok{11.6}\NormalTok{, }\FloatTok{12.4}\NormalTok{, }\FloatTok{61.5}\NormalTok{)}\SpecialCharTok{/}\DecValTok{100}
  \CommentTok{\# Prior covariance of returns}
\NormalTok{  Sigma }\OtherTok{=}\NormalTok{ Corrmat }\SpecialCharTok{*}\NormalTok{ (stdevs }\SpecialCharTok{\%*\%} \FunctionTok{t}\NormalTok{(stdevs))}
\end{Highlighting}
\end{Shaded}

Equilibrium risk premium

\begin{Shaded}
\begin{Highlighting}[]
\CommentTok{\# risk aversion parameter}
\NormalTok{delta }\OtherTok{=} \FloatTok{2.5}
\NormalTok{Pi }\OtherTok{=}\NormalTok{ delta }\SpecialCharTok{*}\NormalTok{ Sigma }\SpecialCharTok{\%*\%}\NormalTok{ w.eq}
\end{Highlighting}
\end{Shaded}

Summary market data

\begin{tabular}{llll}
\toprule
Assets & Std Dev & Weq & PI\\
\midrule
Australia & 16 & 1.6 & 3.9\\
Canada & 20.3 & 2.2 & 6.9\\
France & 24.8 & 5.2 & 8.4\\
Germany & 27.1 & 5.5 & 9\\
Japan & 21 & 11.6 & 4.3\\
\addlinespace
UK & 20 & 12.4 & 6.8\\
USA & 18.7 & 61.5 & 7.6\\
\bottomrule
\end{tabular}

\hypertarget{view-1-is-the-german-equity-market-will-outperform-the-rest-of-european-markets-by-5-a-year.}{%
\subsection{View 1: is The German equity market will outperform the rest
of European Markets by 5\% a
year.}\label{view-1-is-the-german-equity-market-will-outperform-the-rest-of-european-markets-by-5-a-year.}}

\begin{Shaded}
\begin{Highlighting}[]
\NormalTok{P }\OtherTok{=} \FunctionTok{matrix}\NormalTok{(}\FunctionTok{c}\NormalTok{(}\DecValTok{0}\NormalTok{, }\DecValTok{0}\NormalTok{, }\SpecialCharTok{{-}}\FloatTok{29.5}\NormalTok{, }\DecValTok{100}\NormalTok{, }\DecValTok{0}\NormalTok{, }\SpecialCharTok{{-}}\FloatTok{70.5}\NormalTok{, }\DecValTok{0}\NormalTok{)}\SpecialCharTok{/}\DecValTok{100}\NormalTok{, }\AttributeTok{nrow=}\DecValTok{1}\NormalTok{)}
\NormalTok{Q }\OtherTok{=} \DecValTok{5}\SpecialCharTok{/}\DecValTok{100}
\NormalTok{tau }\OtherTok{=} \FloatTok{0.05}

\NormalTok{Omega }\OtherTok{=} \FunctionTok{as.matrix}\NormalTok{(}\FunctionTok{diag}\NormalTok{(tau }\SpecialCharTok{*}\NormalTok{ P }\SpecialCharTok{\%*\%}\NormalTok{ Sigma }\SpecialCharTok{\%*\%} \FunctionTok{t}\NormalTok{(P)))}
\NormalTok{tau.Sigma.inv }\OtherTok{=} \FunctionTok{solve}\NormalTok{(tau}\SpecialCharTok{*}\NormalTok{Sigma)}
\NormalTok{M.inverse }\OtherTok{=} \FunctionTok{solve}\NormalTok{(tau.Sigma.inv }\SpecialCharTok{+}\NormalTok{ (}\FunctionTok{t}\NormalTok{(P) }\SpecialCharTok{\%*\%} \FunctionTok{solve}\NormalTok{(Omega) }\SpecialCharTok{\%*\%}\NormalTok{ P))}
\NormalTok{mu.bar }\OtherTok{=}\NormalTok{ M.inverse }\SpecialCharTok{\%*\%}\NormalTok{ (tau.Sigma.inv }\SpecialCharTok{\%*\%}\NormalTok{ Pi }\SpecialCharTok{+} \FunctionTok{t}\NormalTok{(P) }\SpecialCharTok{\%*\%} \FunctionTok{solve}\NormalTok{(Omega) }\SpecialCharTok{\%*\%}\NormalTok{ Q)}
\NormalTok{Sigma.bar }\OtherTok{=}\NormalTok{ M.inverse }\SpecialCharTok{+}\NormalTok{ Sigma}

\NormalTok{w.star }\OtherTok{=}\NormalTok{ (}\DecValTok{1}\SpecialCharTok{/}\NormalTok{delta) }\SpecialCharTok{*} \FunctionTok{solve}\NormalTok{(Sigma.bar) }\SpecialCharTok{\%*\%}\NormalTok{ mu.bar}

\NormalTok{df }\OtherTok{=} \FunctionTok{data.frame}\NormalTok{(}\DecValTok{100}\SpecialCharTok{*}\FunctionTok{cbind}\NormalTok{(}\FunctionTok{t}\NormalTok{(P), mu.bar, w.star, w.star}\SpecialCharTok{{-}}\NormalTok{w.eq}\SpecialCharTok{/}\NormalTok{(}\DecValTok{1}\SpecialCharTok{+}\NormalTok{tau)))}
\FunctionTok{row.names}\NormalTok{(df) }\OtherTok{=}\NormalTok{ AssetNames}
\FunctionTok{names}\NormalTok{(df) }\OtherTok{=}  \FunctionTok{c}\NormalTok{(}\StringTok{\textquotesingle{}P\textquotesingle{}}\NormalTok{, }\StringTok{"$}\SpecialCharTok{\textbackslash{}\textbackslash{}}\StringTok{bar\{}\SpecialCharTok{\textbackslash{}\textbackslash{}}\StringTok{mu\}$"}\NormalTok{, }\StringTok{\textquotesingle{}$w\^{}*$\textquotesingle{}}\NormalTok{,}\StringTok{\textquotesingle{}$w\^{}* {-} }\SpecialCharTok{\textbackslash{}\textbackslash{}}\StringTok{frac\{W\_\{eq\}\}\{1+}\SpecialCharTok{\textbackslash{}\textbackslash{}}\StringTok{tau\}$\textquotesingle{}}\NormalTok{)}
\FunctionTok{kable}\NormalTok{(df, }\AttributeTok{digits =} \DecValTok{1}\NormalTok{, }\AttributeTok{format=}\StringTok{"latex"}\NormalTok{, }\AttributeTok{booktabs=}\NormalTok{T, }\AttributeTok{escape=}\NormalTok{F,}
      \AttributeTok{caption=}\StringTok{"Solution with View 1. P: view matrix, $}\SpecialCharTok{\textbackslash{}\textbackslash{}}\StringTok{bar\{}\SpecialCharTok{\textbackslash{}\textbackslash{}}\StringTok{mu\}$: ex{-}post expected return,}
\StringTok{      $w\^{}*$: optimal weights, $}\SpecialCharTok{\textbackslash{}\textbackslash{}}\StringTok{frac\{W\_\{eq\}\}\{1+}\SpecialCharTok{\textbackslash{}\textbackslash{}}\StringTok{tau\}$: scaled equilibrium weights"}\NormalTok{)}
\end{Highlighting}
\end{Shaded}

\begin{table}

\caption{\label{tab:unnamed-chunk-5}Solution with View 1. P: view matrix, $\bar{\mu}$: ex-post expected return,
      $w^*$: optimal weights, $\frac{W_{eq}}{1+\tau}$: scaled equilibrium weights}
\centering
\begin{tabular}[t]{lrrrr}
\toprule
  & P & $\bar{\mu}$ & $w^*$ & $w^* - \frac{W_{eq}}{1+\tau}$\\
\midrule
Australia & 0.0 & 4.3 & 1.5 & 0.0\\
Canada & 0.0 & 7.6 & 2.1 & 0.0\\
France & -29.5 & 9.3 & -3.9 & -8.9\\
Germany & 100.0 & 11.0 & 35.4 & 30.2\\
Japan & 0.0 & 4.5 & 11.0 & 0.0\\
\addlinespace
UK & -70.5 & 7.0 & -9.5 & -21.3\\
USA & 0.0 & 8.1 & 58.6 & 0.0\\
\bottomrule
\end{tabular}
\end{table}

\end{document}
