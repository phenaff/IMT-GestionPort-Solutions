% Options for packages loaded elsewhere
\PassOptionsToPackage{unicode}{hyperref}
\PassOptionsToPackage{hyphens}{url}
%
\documentclass[
]{article}
\usepackage{amsmath,amssymb}
\usepackage{lmodern}
\usepackage{iftex}
\ifPDFTeX
  \usepackage[T1]{fontenc}
  \usepackage[utf8]{inputenc}
  \usepackage{textcomp} % provide euro and other symbols
\else % if luatex or xetex
  \usepackage{unicode-math}
  \defaultfontfeatures{Scale=MatchLowercase}
  \defaultfontfeatures[\rmfamily]{Ligatures=TeX,Scale=1}
\fi
% Use upquote if available, for straight quotes in verbatim environments
\IfFileExists{upquote.sty}{\usepackage{upquote}}{}
\IfFileExists{microtype.sty}{% use microtype if available
  \usepackage[]{microtype}
  \UseMicrotypeSet[protrusion]{basicmath} % disable protrusion for tt fonts
}{}
\makeatletter
\@ifundefined{KOMAClassName}{% if non-KOMA class
  \IfFileExists{parskip.sty}{%
    \usepackage{parskip}
  }{% else
    \setlength{\parindent}{0pt}
    \setlength{\parskip}{6pt plus 2pt minus 1pt}}
}{% if KOMA class
  \KOMAoptions{parskip=half}}
\makeatother
\usepackage{xcolor}
\IfFileExists{xurl.sty}{\usepackage{xurl}}{} % add URL line breaks if available
\IfFileExists{bookmark.sty}{\usepackage{bookmark}}{\usepackage{hyperref}}
\hypersetup{
  pdftitle={Gestion de Portefeuille},
  pdfauthor={Patrick Hénaff},
  hidelinks,
  pdfcreator={LaTeX via pandoc}}
\urlstyle{same} % disable monospaced font for URLs
\usepackage[margin=1in]{geometry}
\usepackage{color}
\usepackage{fancyvrb}
\newcommand{\VerbBar}{|}
\newcommand{\VERB}{\Verb[commandchars=\\\{\}]}
\DefineVerbatimEnvironment{Highlighting}{Verbatim}{commandchars=\\\{\}}
% Add ',fontsize=\small' for more characters per line
\usepackage{framed}
\definecolor{shadecolor}{RGB}{248,248,248}
\newenvironment{Shaded}{\begin{snugshade}}{\end{snugshade}}
\newcommand{\AlertTok}[1]{\textcolor[rgb]{0.94,0.16,0.16}{#1}}
\newcommand{\AnnotationTok}[1]{\textcolor[rgb]{0.56,0.35,0.01}{\textbf{\textit{#1}}}}
\newcommand{\AttributeTok}[1]{\textcolor[rgb]{0.77,0.63,0.00}{#1}}
\newcommand{\BaseNTok}[1]{\textcolor[rgb]{0.00,0.00,0.81}{#1}}
\newcommand{\BuiltInTok}[1]{#1}
\newcommand{\CharTok}[1]{\textcolor[rgb]{0.31,0.60,0.02}{#1}}
\newcommand{\CommentTok}[1]{\textcolor[rgb]{0.56,0.35,0.01}{\textit{#1}}}
\newcommand{\CommentVarTok}[1]{\textcolor[rgb]{0.56,0.35,0.01}{\textbf{\textit{#1}}}}
\newcommand{\ConstantTok}[1]{\textcolor[rgb]{0.00,0.00,0.00}{#1}}
\newcommand{\ControlFlowTok}[1]{\textcolor[rgb]{0.13,0.29,0.53}{\textbf{#1}}}
\newcommand{\DataTypeTok}[1]{\textcolor[rgb]{0.13,0.29,0.53}{#1}}
\newcommand{\DecValTok}[1]{\textcolor[rgb]{0.00,0.00,0.81}{#1}}
\newcommand{\DocumentationTok}[1]{\textcolor[rgb]{0.56,0.35,0.01}{\textbf{\textit{#1}}}}
\newcommand{\ErrorTok}[1]{\textcolor[rgb]{0.64,0.00,0.00}{\textbf{#1}}}
\newcommand{\ExtensionTok}[1]{#1}
\newcommand{\FloatTok}[1]{\textcolor[rgb]{0.00,0.00,0.81}{#1}}
\newcommand{\FunctionTok}[1]{\textcolor[rgb]{0.00,0.00,0.00}{#1}}
\newcommand{\ImportTok}[1]{#1}
\newcommand{\InformationTok}[1]{\textcolor[rgb]{0.56,0.35,0.01}{\textbf{\textit{#1}}}}
\newcommand{\KeywordTok}[1]{\textcolor[rgb]{0.13,0.29,0.53}{\textbf{#1}}}
\newcommand{\NormalTok}[1]{#1}
\newcommand{\OperatorTok}[1]{\textcolor[rgb]{0.81,0.36,0.00}{\textbf{#1}}}
\newcommand{\OtherTok}[1]{\textcolor[rgb]{0.56,0.35,0.01}{#1}}
\newcommand{\PreprocessorTok}[1]{\textcolor[rgb]{0.56,0.35,0.01}{\textit{#1}}}
\newcommand{\RegionMarkerTok}[1]{#1}
\newcommand{\SpecialCharTok}[1]{\textcolor[rgb]{0.00,0.00,0.00}{#1}}
\newcommand{\SpecialStringTok}[1]{\textcolor[rgb]{0.31,0.60,0.02}{#1}}
\newcommand{\StringTok}[1]{\textcolor[rgb]{0.31,0.60,0.02}{#1}}
\newcommand{\VariableTok}[1]{\textcolor[rgb]{0.00,0.00,0.00}{#1}}
\newcommand{\VerbatimStringTok}[1]{\textcolor[rgb]{0.31,0.60,0.02}{#1}}
\newcommand{\WarningTok}[1]{\textcolor[rgb]{0.56,0.35,0.01}{\textbf{\textit{#1}}}}
\usepackage{graphicx}
\makeatletter
\def\maxwidth{\ifdim\Gin@nat@width>\linewidth\linewidth\else\Gin@nat@width\fi}
\def\maxheight{\ifdim\Gin@nat@height>\textheight\textheight\else\Gin@nat@height\fi}
\makeatother
% Scale images if necessary, so that they will not overflow the page
% margins by default, and it is still possible to overwrite the defaults
% using explicit options in \includegraphics[width, height, ...]{}
\setkeys{Gin}{width=\maxwidth,height=\maxheight,keepaspectratio}
% Set default figure placement to htbp
\makeatletter
\def\fps@figure{htbp}
\makeatother
\setlength{\emergencystretch}{3em} % prevent overfull lines
\providecommand{\tightlist}{%
  \setlength{\itemsep}{0pt}\setlength{\parskip}{0pt}}
\setcounter{secnumdepth}{-\maxdimen} % remove section numbering
\usepackage[utf8]{inputenc}
\usepackage{booktabs}
\usepackage{longtable}
\usepackage{array}
\usepackage{multirow}
\usepackage{wrapfig}
\usepackage{float}
\usepackage{colortbl}
\usepackage{pdflscape}
\usepackage{tabu}
\usepackage{threeparttable}
\usepackage{threeparttablex}
\usepackage[normalem]{ulem}
\usepackage{makecell}
\usepackage{xcolor}
\ifLuaTeX
  \usepackage{selnolig}  % disable illegal ligatures
\fi

\title{Gestion de Portefeuille}
\usepackage{etoolbox}
\makeatletter
\providecommand{\subtitle}[1]{% add subtitle to \maketitle
  \apptocmd{\@title}{\par {\large #1 \par}}{}{}
}
\makeatother
\subtitle{TP-3: Modèle à un facteur et modèle de Treynor Black}
\author{Patrick Hénaff}
\date{Février-Mars 2020}

\begin{document}
\maketitle

\begin{Shaded}
\begin{Highlighting}[]
\FunctionTok{library}\NormalTok{(xts)}
\FunctionTok{library}\NormalTok{(hornpa)}
\FunctionTok{library}\NormalTok{(lubridate)}
\FunctionTok{library}\NormalTok{(xtable)}
\FunctionTok{library}\NormalTok{(PerformanceAnalytics)}
\FunctionTok{library}\NormalTok{(TTR)}
\FunctionTok{library}\NormalTok{(lubridate)}
\FunctionTok{library}\NormalTok{(roll)}
\FunctionTok{library}\NormalTok{(Hmisc)}
\FunctionTok{library}\NormalTok{(nFactors)}
\FunctionTok{library}\NormalTok{(kableExtra)}
\FunctionTok{library}\NormalTok{(broom)}
\FunctionTok{library}\NormalTok{(quadprog)}
\end{Highlighting}
\end{Shaded}

\hypertarget{donnuxe9es}{%
\section{Données}\label{donnuxe9es}}

\hypertarget{suxe9ries-de-rendement-mensuel-pour-11-valeurs}{%
\subsection{Séries de rendement mensuel pour 11
valeurs:}\label{suxe9ries-de-rendement-mensuel-pour-11-valeurs}}

\begin{Shaded}
\begin{Highlighting}[]
\NormalTok{monthly.ret.file }\OtherTok{\textless{}{-}} \StringTok{"./monthly.ret.rda"}
\FunctionTok{load}\NormalTok{(monthly.ret.file)}
\FunctionTok{index}\NormalTok{(monthly.ret) }\OtherTok{\textless{}{-}} \FunctionTok{floor\_date}\NormalTok{(}\FunctionTok{index}\NormalTok{(monthly.ret), }\StringTok{"month"}\NormalTok{)}
\end{Highlighting}
\end{Shaded}

\hypertarget{matrice-de-covariance-des-rendements}{%
\subsection{Matrice de covariance des
rendements:}\label{matrice-de-covariance-des-rendements}}

\begin{Shaded}
\begin{Highlighting}[]
\FunctionTok{kable}\NormalTok{(}\FunctionTok{cov}\NormalTok{(monthly.ret), }\StringTok{"latex"}\NormalTok{, }\AttributeTok{booktabs=}\NormalTok{T) }\SpecialCharTok{\%\textgreater{}\%}
\FunctionTok{kable\_styling}\NormalTok{(}\AttributeTok{latex\_options=}\FunctionTok{c}\NormalTok{(}\StringTok{"scale\_down"}\NormalTok{, }\StringTok{"HOLD\_position"}\NormalTok{))}
\end{Highlighting}
\end{Shaded}

\begin{table}[H]
\centering
\resizebox{\linewidth}{!}{
\begin{tabular}{lrrrrrrrrrrr}
\toprule
  & AAPL & AMZN & MSFT & F & SPY & QQQ & XOM & MMM & HD & PG & KO\\
\midrule
AAPL & 0.0079015 & 0.0035933 & 0.0028724 & 0.0036506 & 0.0021193 & 0.0033242 & 0.0012183 & 0.0019158 & 0.0012159 & 0.0009073 & 0.0009576\\
AMZN & 0.0035933 & 0.0097937 & 0.0026625 & 0.0025940 & 0.0020258 & 0.0030033 & 0.0011468 & 0.0016726 & 0.0016066 & 0.0003831 & 0.0013968\\
MSFT & 0.0028724 & 0.0026625 & 0.0044949 & 0.0032132 & 0.0017774 & 0.0022969 & 0.0009976 & 0.0012898 & 0.0015753 & 0.0007414 & 0.0011363\\
F & 0.0036506 & 0.0025940 & 0.0032132 & 0.0226257 & 0.0032869 & 0.0034954 & 0.0017697 & 0.0034663 & 0.0032642 & 0.0014660 & 0.0014993\\
SPY & 0.0021193 & 0.0020258 & 0.0017774 & 0.0032869 & 0.0017549 & 0.0019207 & 0.0012159 & 0.0016906 & 0.0015105 & 0.0008284 & 0.0009008\\
\addlinespace
QQQ & 0.0033242 & 0.0030033 & 0.0022969 & 0.0034954 & 0.0019207 & 0.0025159 & 0.0010479 & 0.0016973 & 0.0016125 & 0.0007561 & 0.0008650\\
XOM & 0.0012183 & 0.0011468 & 0.0009976 & 0.0017697 & 0.0012159 & 0.0010479 & 0.0025213 & 0.0015076 & 0.0008121 & 0.0006409 & 0.0007365\\
MMM & 0.0019158 & 0.0016726 & 0.0012898 & 0.0034663 & 0.0016906 & 0.0016973 & 0.0015076 & 0.0032027 & 0.0016559 & 0.0009968 & 0.0008642\\
HD & 0.0012159 & 0.0016066 & 0.0015753 & 0.0032642 & 0.0015105 & 0.0016125 & 0.0008121 & 0.0016559 & 0.0037458 & 0.0005615 & 0.0005566\\
PG & 0.0009073 & 0.0003831 & 0.0007414 & 0.0014660 & 0.0008284 & 0.0007561 & 0.0006409 & 0.0009968 & 0.0005615 & 0.0018508 & 0.0009004\\
\addlinespace
KO & 0.0009576 & 0.0013968 & 0.0011363 & 0.0014993 & 0.0009008 & 0.0008650 & 0.0007365 & 0.0008642 & 0.0005566 & 0.0009004 & 0.0019550\\
\bottomrule
\end{tabular}}
\end{table}

\hypertarget{rendement-moyen-mensuel}{%
\subsection{Rendement moyen mensuel}\label{rendement-moyen-mensuel}}

\begin{Shaded}
\begin{Highlighting}[]
\FunctionTok{kbl}\NormalTok{(}\FunctionTok{colMeans}\NormalTok{(monthly.ret), }\AttributeTok{format=}\StringTok{"latex"}\NormalTok{, }\AttributeTok{booktabs=}\NormalTok{T, }
    \AttributeTok{col.names=}\FunctionTok{c}\NormalTok{(}\StringTok{"Rendement"}\NormalTok{), }\AttributeTok{caption=}\StringTok{"Rendement moyen mensuel"}\NormalTok{) }\SpecialCharTok{\%\textgreater{}\%}
    \FunctionTok{kable\_styling}\NormalTok{(}\AttributeTok{latex\_options=}\StringTok{"HOLD\_position"}\NormalTok{)}
\end{Highlighting}
\end{Shaded}

\begin{table}[H]

\caption{\label{tab:unnamed-chunk-3}Rendement moyen mensuel}
\centering
\begin{tabular}[t]{lr}
\toprule
  & Rendement\\
\midrule
AAPL & 0.0254037\\
AMZN & 0.0298355\\
MSFT & 0.0151864\\
F & 0.0115177\\
SPY & 0.0075856\\
\addlinespace
QQQ & 0.0122593\\
XOM & 0.0016595\\
MMM & 0.0079299\\
HD & 0.0151356\\
PG & 0.0073821\\
\addlinespace
KO & 0.0100164\\
\bottomrule
\end{tabular}
\end{table}

\hypertarget{taux-sans-risque}{%
\subsection{Taux sans risque}\label{taux-sans-risque}}

Le taux sans risque mensuel est obtenu de la Réserve Fédérale US. A
diviser par 12 pour être cohérent avec les rendement des titres.

\begin{Shaded}
\begin{Highlighting}[]
\NormalTok{tmp }\OtherTok{\textless{}{-}} \FunctionTok{read.csv}\NormalTok{(}\StringTok{"DP\_LIVE\_01032020211755676.csv"}\NormalTok{, }\AttributeTok{header=}\ConstantTok{TRUE}\NormalTok{, }\AttributeTok{sep=}\StringTok{";"}\NormalTok{)[, }\FunctionTok{c}\NormalTok{(}\StringTok{"TIME"}\NormalTok{, }\StringTok{"Value"}\NormalTok{)]}
\NormalTok{dt }\OtherTok{\textless{}{-}} \FunctionTok{ymd}\NormalTok{(}\FunctionTok{paste}\NormalTok{(tmp}\SpecialCharTok{$}\NormalTok{TIME, }\StringTok{"{-}01"}\NormalTok{, }\AttributeTok{sep=}\StringTok{""}\NormalTok{))}
\NormalTok{rf\_rate }\OtherTok{\textless{}{-}} \FunctionTok{xts}\NormalTok{((tmp}\SpecialCharTok{$}\NormalTok{Value}\SpecialCharTok{/}\FloatTok{100.0}\NormalTok{)}\SpecialCharTok{/}\DecValTok{12}\NormalTok{, dt)}
\FunctionTok{colnames}\NormalTok{(rf\_rate) }\OtherTok{\textless{}{-}} \StringTok{"Rf"}
\NormalTok{monthly.ret}\FloatTok{.2} \OtherTok{\textless{}{-}} \FunctionTok{merge.xts}\NormalTok{(monthly.ret, rf\_rate, }\AttributeTok{join=}\StringTok{"inner"}\NormalTok{)}
\end{Highlighting}
\end{Shaded}

\begin{figure}
\centering
\includegraphics{TP-3_files/figure-latex/unnamed-chunk-5-1.pdf}
\caption{taux sans risque mensuel}
\end{figure}

\hypertarget{estimation-dun-moduxe8le-uxe0-un-facteur}{%
\section{Estimation d'un modèle à un
facteur}\label{estimation-dun-moduxe8le-uxe0-un-facteur}}

\begin{itemize}
\tightlist
\item
  Utiliser l'indice SPY comme proxy pour le marché et estimer pour
  chaque titre le modèle:
\end{itemize}

\[
R_i(t) - R_f(t) = \alpha + \beta (R_M(t) - R_f(t)) + \epsilon(t)
\] en utilisant la fonction \texttt{lm}. - Placer chaque titre sur un
diagramme rendement/beta et calculer par regression la droite de marché
des titres risqués. - En déduire les titres qui, selon ce modèle,
\emph{semblent} chers et ceux qui semblent sous-évalués.

Est-ce que ces mesures de cherté relative vous semble correctes? Essayez
de mesurer la robustesse de ce calcul en estimant le modèles sur des
sous-intervalles de temps.

Présentez vos résultats de manière synthétique.

\hypertarget{moduxe8le-de-treynor-black}{%
\section{Modèle de Treynor-Black}\label{moduxe8le-de-treynor-black}}

Le modèle de Treynor-Black a pour objectif d'exploiter les informations
calculées en première partie. L'idée étant de constituer un portefeuille
``actif'' avec les titres qui semblent mal valorisés par le marché, et
allouer le reste de sa richesse au portefeuille de marché.

\hypertarget{selection-des-titres-uxe0-inclure-dans-le-portefeuille-actif.}{%
\subsection{Selection des titres à inclure dans le portefeuille
actif.}\label{selection-des-titres-uxe0-inclure-dans-le-portefeuille-actif.}}

C'est l'étape délicate de la méthode de Treynor-Black. A partir de
l'évaluation du modèle à un facteur, déterminez quels titres méritent de
figurer dans le portefeuille actif. En théorie, on a envie d'acheter les
titres sous-cotés (\(\alpha_i > 0\)) mais cette anomalie n'est peut être
qu'apparente! Il faut également apprécier la qualité de l'estimation
statistique.

En testant diverses combinaisons de titres à mettre dans le portefeuille
actif, vous pourrez mesurer la sensibilité de modèle de Treynor-Black
aux données.

\hypertarget{duxe9termination-du-portefeuille-actif}{%
\subsection{Détermination du portefeuille
actif}\label{duxe9termination-du-portefeuille-actif}}

Ayant choisi les titres à inclure dans le portefeuille actif, on
rappelle que le poids de chaque titre dans le portefeuille actif est
proportionnel au ratio \(\alpha_i/\sigma^2(\epsilon_i)\):

\[
w_i = \frac{\alpha_i/\sigma^2(\epsilon_i)}{\sum_i \alpha_i/\sigma^2(\epsilon_i)}
\]

Calculer les poids des actifs dans le portefeuille actif. Justifier
votre choix d'inclure ou d'exclure tel ou tel instrument.

Calculez les valeurs suivantes concernant le portefeuille actif:

\begin{description}
\item[$R_A$] Excess de rendement
\item[$\alpha_A$] alpha du portefeuille actif
\item[$\beta_A$]  beta du portefeuille actif
\item[$\sigma_A$] ecart-type du portefeuille actif
\item[$\sigma^2(e_A)$] variance résiduelle du portefeuille actif

\end{description}

\hypertarget{duxe9termination-de-la-ponduxe9ration-entre-le-portefeuille-actif-et-le-portefeuille-de-marchuxe9.}{%
\subsection{Détermination de la pondération entre le portefeuille actif
et le portefeuille de
marché.}\label{duxe9termination-de-la-ponduxe9ration-entre-le-portefeuille-actif-et-le-portefeuille-de-marchuxe9.}}

On rappelle l'allocation de richesse au portefeuille actif:

\[
w_A = \frac{\alpha_A \sigma^2_M}{\alpha_A \sigma^2_M (1-\beta_A) + R_M \sigma^2(e_A)}
\]

Avec:

\[
\begin{aligned}
R_A & = \alpha_A + \beta_A R_M \\
\sigma^2_A & = \beta^2_A \sigma^2_M + \sigma^2(e_A)
\end{aligned}
\]

\hypertarget{capital-allocation-line}{%
\subsection{Capital Allocation Line}\label{capital-allocation-line}}

Calculez l'espérance de rendement et le risque de quelques portefeuilles
situés sur la ``Capital Allocation Line'' qui joint l'actif sans risque
et le portefeuille tangent. Placez la solution du modèle de
Treynor-Black, le portefeuille actif et le portefeuille de marché sur le
graphique ci-dessous.

\begin{Shaded}
\begin{Highlighting}[]
\NormalTok{Assets }\OtherTok{\textless{}{-}} \FunctionTok{c}\NormalTok{(}\StringTok{"AAPL"}\NormalTok{, }\StringTok{"AMZN"}\NormalTok{, }\StringTok{"MSFT"}\NormalTok{, }\StringTok{"F"}\NormalTok{,  }\StringTok{"XOM"}\NormalTok{, }\StringTok{"MMM"}\NormalTok{,  }\StringTok{"HD"}\NormalTok{,   }\StringTok{"PG"}\NormalTok{,   }\StringTok{"KO"}\NormalTok{)}
\NormalTok{plot.data }\OtherTok{\textless{}{-}}\NormalTok{ monthly.ret}\FloatTok{.2}\NormalTok{[, }\FunctionTok{c}\NormalTok{(Assets, }\StringTok{"Rf"}\NormalTok{)]}
\ControlFlowTok{for}\NormalTok{(a }\ControlFlowTok{in}\NormalTok{ Assets) \{}
\NormalTok{  plot.data[, a] }\OtherTok{\textless{}{-}}\NormalTok{ plot.data[, a] }\SpecialCharTok{{-}}\NormalTok{ plot.data}\SpecialCharTok{$}\NormalTok{Rf}
\NormalTok{  \}}

\NormalTok{res }\OtherTok{\textless{}{-}} \FunctionTok{data.frame}\NormalTok{(}\AttributeTok{Mean=}\FunctionTok{apply}\NormalTok{(plot.data[, Assets],}\DecValTok{2}\NormalTok{,mean),}
                  \AttributeTok{Sd =} \FunctionTok{apply}\NormalTok{(plot.data[, Assets],}\DecValTok{2}\NormalTok{,sd))}
\FunctionTok{rownames}\NormalTok{(res) }\OtherTok{\textless{}{-}}\NormalTok{ Assets}
\end{Highlighting}
\end{Shaded}

\begin{Shaded}
\begin{Highlighting}[]
\FunctionTok{plot}\NormalTok{(Mean }\SpecialCharTok{\textasciitilde{}}\NormalTok{ Sd, }\AttributeTok{data=}\NormalTok{res, }\AttributeTok{xlim=}\FunctionTok{c}\NormalTok{(}\DecValTok{0}\NormalTok{, }\FloatTok{0.4}\NormalTok{), }\AttributeTok{ylim=}\FunctionTok{c}\NormalTok{(}\DecValTok{0}\NormalTok{, .}\DecValTok{05}\NormalTok{), }\AttributeTok{xlab=}\FunctionTok{expression}\NormalTok{(sigma),}
     \AttributeTok{ylab=}\StringTok{"Excess Return"}\NormalTok{, }\AttributeTok{cex=}\NormalTok{.}\DecValTok{5}\NormalTok{, }\AttributeTok{bty=}\StringTok{"n"}\NormalTok{, }\AttributeTok{cex.lab=}\DecValTok{1}\NormalTok{)}
\FunctionTok{with}\NormalTok{(res, }\FunctionTok{text}\NormalTok{(Mean }\SpecialCharTok{\textasciitilde{}}\NormalTok{ Sd, }\AttributeTok{labels=}\FunctionTok{row.names}\NormalTok{(res), }\AttributeTok{pos=}\DecValTok{4}\NormalTok{, }\AttributeTok{cex=}\FloatTok{0.7}\NormalTok{, }\AttributeTok{col=}\StringTok{"blue"}\NormalTok{))}
\end{Highlighting}
\end{Shaded}

\includegraphics{TP-3_files/figure-latex/unnamed-chunk-7-1.pdf}

\end{document}
