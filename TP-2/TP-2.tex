% Options for packages loaded elsewhere
\PassOptionsToPackage{unicode}{hyperref}
\PassOptionsToPackage{hyphens}{url}
%
\documentclass[
]{article}
\usepackage{amsmath,amssymb}
\usepackage{lmodern}
\usepackage{iftex}
\ifPDFTeX
  \usepackage[T1]{fontenc}
  \usepackage[utf8]{inputenc}
  \usepackage{textcomp} % provide euro and other symbols
\else % if luatex or xetex
  \usepackage{unicode-math}
  \defaultfontfeatures{Scale=MatchLowercase}
  \defaultfontfeatures[\rmfamily]{Ligatures=TeX,Scale=1}
\fi
% Use upquote if available, for straight quotes in verbatim environments
\IfFileExists{upquote.sty}{\usepackage{upquote}}{}
\IfFileExists{microtype.sty}{% use microtype if available
  \usepackage[]{microtype}
  \UseMicrotypeSet[protrusion]{basicmath} % disable protrusion for tt fonts
}{}
\makeatletter
\@ifundefined{KOMAClassName}{% if non-KOMA class
  \IfFileExists{parskip.sty}{%
    \usepackage{parskip}
  }{% else
    \setlength{\parindent}{0pt}
    \setlength{\parskip}{6pt plus 2pt minus 1pt}}
}{% if KOMA class
  \KOMAoptions{parskip=half}}
\makeatother
\usepackage{xcolor}
\IfFileExists{xurl.sty}{\usepackage{xurl}}{} % add URL line breaks if available
\IfFileExists{bookmark.sty}{\usepackage{bookmark}}{\usepackage{hyperref}}
\hypersetup{
  pdftitle={Gestion de Portefeuille},
  pdfauthor={Patrick Hénaff},
  hidelinks,
  pdfcreator={LaTeX via pandoc}}
\urlstyle{same} % disable monospaced font for URLs
\usepackage[margin=1in]{geometry}
\usepackage{color}
\usepackage{fancyvrb}
\newcommand{\VerbBar}{|}
\newcommand{\VERB}{\Verb[commandchars=\\\{\}]}
\DefineVerbatimEnvironment{Highlighting}{Verbatim}{commandchars=\\\{\}}
% Add ',fontsize=\small' for more characters per line
\usepackage{framed}
\definecolor{shadecolor}{RGB}{248,248,248}
\newenvironment{Shaded}{\begin{snugshade}}{\end{snugshade}}
\newcommand{\AlertTok}[1]{\textcolor[rgb]{0.94,0.16,0.16}{#1}}
\newcommand{\AnnotationTok}[1]{\textcolor[rgb]{0.56,0.35,0.01}{\textbf{\textit{#1}}}}
\newcommand{\AttributeTok}[1]{\textcolor[rgb]{0.77,0.63,0.00}{#1}}
\newcommand{\BaseNTok}[1]{\textcolor[rgb]{0.00,0.00,0.81}{#1}}
\newcommand{\BuiltInTok}[1]{#1}
\newcommand{\CharTok}[1]{\textcolor[rgb]{0.31,0.60,0.02}{#1}}
\newcommand{\CommentTok}[1]{\textcolor[rgb]{0.56,0.35,0.01}{\textit{#1}}}
\newcommand{\CommentVarTok}[1]{\textcolor[rgb]{0.56,0.35,0.01}{\textbf{\textit{#1}}}}
\newcommand{\ConstantTok}[1]{\textcolor[rgb]{0.00,0.00,0.00}{#1}}
\newcommand{\ControlFlowTok}[1]{\textcolor[rgb]{0.13,0.29,0.53}{\textbf{#1}}}
\newcommand{\DataTypeTok}[1]{\textcolor[rgb]{0.13,0.29,0.53}{#1}}
\newcommand{\DecValTok}[1]{\textcolor[rgb]{0.00,0.00,0.81}{#1}}
\newcommand{\DocumentationTok}[1]{\textcolor[rgb]{0.56,0.35,0.01}{\textbf{\textit{#1}}}}
\newcommand{\ErrorTok}[1]{\textcolor[rgb]{0.64,0.00,0.00}{\textbf{#1}}}
\newcommand{\ExtensionTok}[1]{#1}
\newcommand{\FloatTok}[1]{\textcolor[rgb]{0.00,0.00,0.81}{#1}}
\newcommand{\FunctionTok}[1]{\textcolor[rgb]{0.00,0.00,0.00}{#1}}
\newcommand{\ImportTok}[1]{#1}
\newcommand{\InformationTok}[1]{\textcolor[rgb]{0.56,0.35,0.01}{\textbf{\textit{#1}}}}
\newcommand{\KeywordTok}[1]{\textcolor[rgb]{0.13,0.29,0.53}{\textbf{#1}}}
\newcommand{\NormalTok}[1]{#1}
\newcommand{\OperatorTok}[1]{\textcolor[rgb]{0.81,0.36,0.00}{\textbf{#1}}}
\newcommand{\OtherTok}[1]{\textcolor[rgb]{0.56,0.35,0.01}{#1}}
\newcommand{\PreprocessorTok}[1]{\textcolor[rgb]{0.56,0.35,0.01}{\textit{#1}}}
\newcommand{\RegionMarkerTok}[1]{#1}
\newcommand{\SpecialCharTok}[1]{\textcolor[rgb]{0.00,0.00,0.00}{#1}}
\newcommand{\SpecialStringTok}[1]{\textcolor[rgb]{0.31,0.60,0.02}{#1}}
\newcommand{\StringTok}[1]{\textcolor[rgb]{0.31,0.60,0.02}{#1}}
\newcommand{\VariableTok}[1]{\textcolor[rgb]{0.00,0.00,0.00}{#1}}
\newcommand{\VerbatimStringTok}[1]{\textcolor[rgb]{0.31,0.60,0.02}{#1}}
\newcommand{\WarningTok}[1]{\textcolor[rgb]{0.56,0.35,0.01}{\textbf{\textit{#1}}}}
\usepackage{graphicx}
\makeatletter
\def\maxwidth{\ifdim\Gin@nat@width>\linewidth\linewidth\else\Gin@nat@width\fi}
\def\maxheight{\ifdim\Gin@nat@height>\textheight\textheight\else\Gin@nat@height\fi}
\makeatother
% Scale images if necessary, so that they will not overflow the page
% margins by default, and it is still possible to overwrite the defaults
% using explicit options in \includegraphics[width, height, ...]{}
\setkeys{Gin}{width=\maxwidth,height=\maxheight,keepaspectratio}
% Set default figure placement to htbp
\makeatletter
\def\fps@figure{htbp}
\makeatother
\setlength{\emergencystretch}{3em} % prevent overfull lines
\providecommand{\tightlist}{%
  \setlength{\itemsep}{0pt}\setlength{\parskip}{0pt}}
\setcounter{secnumdepth}{-\maxdimen} % remove section numbering
\usepackage[utf8]{inputenc}
\usepackage{booktabs}
\usepackage{longtable}
\usepackage{array}
\usepackage{multirow}
\usepackage{wrapfig}
\usepackage{float}
\usepackage{colortbl}
\usepackage{pdflscape}
\usepackage{tabu}
\usepackage{threeparttable}
\usepackage{threeparttablex}
\usepackage[normalem]{ulem}
\usepackage{makecell}
\usepackage{xcolor}
\ifLuaTeX
  \usepackage{selnolig}  % disable illegal ligatures
\fi

\title{Gestion de Portefeuille}
\usepackage{etoolbox}
\makeatletter
\providecommand{\subtitle}[1]{% add subtitle to \maketitle
  \apptocmd{\@title}{\par {\large #1 \par}}{}{}
}
\makeatother
\subtitle{TP-2: Droite de Marchés des Capitaux}
\author{Patrick Hénaff}
\date{Février-Mars 2021}

\begin{document}
\maketitle

\hypertarget{donnuxe9es}{%
\section{Données}\label{donnuxe9es}}

\hypertarget{suxe9ries-de-rendement-quatidien-pour-11-valeurs}{%
\subsection{Séries de rendement quatidien pour 11
valeurs:}\label{suxe9ries-de-rendement-quatidien-pour-11-valeurs}}

\begin{Shaded}
\begin{Highlighting}[]
\NormalTok{daily.ret.file }\OtherTok{\textless{}{-}} \FunctionTok{file.path}\NormalTok{(}\FunctionTok{get.data.folder}\NormalTok{(), }\StringTok{"daily.ret.rda"}\NormalTok{)}
\FunctionTok{load}\NormalTok{(daily.ret.file)}
\FunctionTok{kable}\NormalTok{(}\FunctionTok{table.Stats}\NormalTok{(daily.ret), }\StringTok{"latex"}\NormalTok{, }\AttributeTok{booktabs=}\NormalTok{T) }\SpecialCharTok{\%\textgreater{}\%} \FunctionTok{kable\_styling}\NormalTok{(}\AttributeTok{latex\_options=}\StringTok{"scale\_down"}\NormalTok{)}
\end{Highlighting}
\end{Shaded}

\begin{table}[H]
\centering
\resizebox{\linewidth}{!}{
\begin{tabular}{lrrrrrrrrrrr}
\toprule
  & AAPL & AMZN & MSFT & F & SPY & QQQ & XOM & MMM & HD & PG & KO\\
\midrule
Observations & 3308.0000 & 3308.0000 & 3308.0000 & 3308.0000 & 3308.0000 & 3308.0000 & 3308.0000 & 3308.0000 & 3308.0000 & 3308.0000 & 3308.0000\\
NAs & 0.0000 & 0.0000 & 0.0000 & 0.0000 & 0.0000 & 0.0000 & 0.0000 & 0.0000 & 0.0000 & 0.0000 & 0.0000\\
Minimum & -0.1792 & -0.1278 & -0.1171 & -0.2500 & -0.0984 & -0.0896 & -0.1395 & -0.1295 & -0.0822 & -0.0790 & -0.0867\\
Quartile 1 & -0.0077 & -0.0094 & -0.0073 & -0.0103 & -0.0038 & -0.0047 & -0.0068 & -0.0055 & -0.0067 & -0.0046 & -0.0047\\
Median & 0.0010 & 0.0008 & 0.0005 & 0.0000 & 0.0006 & 0.0010 & 0.0001 & 0.0008 & 0.0006 & 0.0004 & 0.0007\\
\addlinespace
Arithmetic Mean & 0.0012 & 0.0015 & 0.0008 & 0.0005 & 0.0004 & 0.0006 & 0.0001 & 0.0004 & 0.0008 & 0.0004 & 0.0005\\
Geometric Mean & 0.0010 & 0.0012 & 0.0006 & 0.0001 & 0.0003 & 0.0005 & 0.0000 & 0.0003 & 0.0006 & 0.0003 & 0.0004\\
Quartile 3 & 0.0112 & 0.0123 & 0.0088 & 0.0106 & 0.0056 & 0.0070 & 0.0073 & 0.0070 & 0.0082 & 0.0055 & 0.0059\\
Maximum & 0.1390 & 0.2695 & 0.1860 & 0.2952 & 0.1452 & 0.1216 & 0.1719 & 0.0988 & 0.1407 & 0.1021 & 0.1388\\
SE Mean & 0.0003 & 0.0004 & 0.0003 & 0.0005 & 0.0002 & 0.0002 & 0.0003 & 0.0002 & 0.0003 & 0.0002 & 0.0002\\
\addlinespace
LCL Mean (0.95) & 0.0005 & 0.0006 & 0.0002 & -0.0005 & 0.0000 & 0.0002 & -0.0004 & -0.0001 & 0.0002 & 0.0000 & 0.0001\\
UCL Mean (0.95) & 0.0019 & 0.0023 & 0.0013 & 0.0014 & 0.0008 & 0.0011 & 0.0006 & 0.0009 & 0.0013 & 0.0007 & 0.0009\\
Variance & 0.0004 & 0.0006 & 0.0003 & 0.0007 & 0.0001 & 0.0002 & 0.0002 & 0.0002 & 0.0003 & 0.0001 & 0.0001\\
Stdev & 0.0196 & 0.0243 & 0.0170 & 0.0266 & 0.0121 & 0.0130 & 0.0150 & 0.0140 & 0.0162 & 0.0109 & 0.0113\\
Skewness & -0.2151 & 1.4889 & 0.4319 & 0.7627 & 0.1379 & -0.0084 & 0.4199 & -0.3815 & 0.5114 & 0.0555 & 0.5004\\
\addlinespace
Kurtosis & 6.2706 & 16.8872 & 10.2176 & 20.9458 & 15.2824 & 7.3976 & 15.4203 & 7.3856 & 6.4641 & 8.1017 & 14.3236\\
\bottomrule
\end{tabular}}
\end{table}

\hypertarget{rendement-annuel-moyen}{%
\subsection{Rendement annuel moyen:}\label{rendement-annuel-moyen}}

\begin{Shaded}
\begin{Highlighting}[]
\FunctionTok{kable}\NormalTok{(}\DecValTok{252}\SpecialCharTok{*}\DecValTok{100}\SpecialCharTok{*}\FunctionTok{colMeans}\NormalTok{(daily.ret), }\StringTok{"latex"}\NormalTok{, }\AttributeTok{booktabs=}\NormalTok{T, }\AttributeTok{digits=}\DecValTok{1}\NormalTok{, }\AttributeTok{col.names=}\FunctionTok{c}\NormalTok{(}\StringTok{"Rendement (\%)"}\NormalTok{), }
      \AttributeTok{caption=}\StringTok{"Rendement annuel moyen"}\NormalTok{)}
\end{Highlighting}
\end{Shaded}

\begin{table}

\caption{\label{tab:unnamed-chunk-2}Rendement annuel moyen}
\centering
\begin{tabular}[t]{lr}
\toprule
  & Rendement (\%)\\
\midrule
AAPL & 30.2\\
AMZN & 37.2\\
MSFT & 19.0\\
F & 11.4\\
SPY & 9.9\\
\addlinespace
QQQ & 15.3\\
XOM & 3.5\\
MMM & 9.9\\
HD & 19.2\\
PG & 9.3\\
\addlinespace
KO & 12.5\\
\bottomrule
\end{tabular}
\end{table}

\hypertarget{matrice-de-corruxe9lation-des-rendements}{%
\subsection{Matrice de corrélation des
rendements:}\label{matrice-de-corruxe9lation-des-rendements}}

\begin{Shaded}
\begin{Highlighting}[]
\NormalTok{correl }\OtherTok{\textless{}{-}} \FunctionTok{cor}\NormalTok{(daily.ret)}
\NormalTok{correl[}\FunctionTok{lower.tri}\NormalTok{(correl)] }\OtherTok{\textless{}{-}} \ConstantTok{NA}
\FunctionTok{options}\NormalTok{(}\AttributeTok{knitr.kable.NA =} \StringTok{\textquotesingle{}\textquotesingle{}}\NormalTok{)}
\FunctionTok{kable}\NormalTok{(correl, }\StringTok{"latex"}\NormalTok{, }\AttributeTok{booktabs=}\NormalTok{T, }\AttributeTok{digits=}\DecValTok{2}\NormalTok{, }\AttributeTok{caption=}\StringTok{"Corrélation des rendements quotidiens"}\NormalTok{) }\SpecialCharTok{\%\textgreater{}\%}
\FunctionTok{kable\_styling}\NormalTok{(}\AttributeTok{latex\_options=}\StringTok{"scale\_down"}\NormalTok{)}
\end{Highlighting}
\end{Shaded}

\begin{table}

\caption{\label{tab:unnamed-chunk-3}Corrélation des rendements quotidiens}
\centering
\resizebox{\linewidth}{!}{
\begin{tabular}[t]{lrrrrrrrrrrr}
\toprule
  & AAPL & AMZN & MSFT & F & SPY & QQQ & XOM & MMM & HD & PG & KO\\
\midrule
AAPL & 1 & 0.46 & 0.49 & 0.37 & 0.61 & 0.75 & 0.40 & 0.45 & 0.42 & 0.32 & 0.32\\
AMZN &  & 1.00 & 0.50 & 0.33 & 0.56 & 0.66 & 0.39 & 0.41 & 0.44 & 0.27 & 0.30\\
MSFT &  &  & 1.00 & 0.39 & 0.71 & 0.76 & 0.53 & 0.53 & 0.49 & 0.44 & 0.46\\
F &  &  &  & 1.00 & 0.56 & 0.53 & 0.37 & 0.44 & 0.46 & 0.30 & 0.31\\
SPY &  &  &  &  & 1.00 & 0.92 & 0.77 & 0.75 & 0.71 & 0.62 & 0.60\\
\addlinespace
QQQ &  &  &  &  &  & 1.00 & 0.64 & 0.69 & 0.66 & 0.52 & 0.52\\
XOM &  &  &  &  &  &  & 1.00 & 0.60 & 0.47 & 0.52 & 0.49\\
MMM &  &  &  &  &  &  &  & 1.00 & 0.55 & 0.50 & 0.47\\
HD &  &  &  &  &  &  &  &  & 1.00 & 0.45 & 0.44\\
PG &  &  &  &  &  &  &  &  &  & 1.00 & 0.57\\
\addlinespace
KO &  &  &  &  &  &  &  &  &  &  & 1.00\\
\bottomrule
\end{tabular}}
\end{table}

\hypertarget{droite-de-marchuxe9-des-capitaux-capital-market-line}{%
\section{Droite de Marché des Capitaux (Capital Market
Line)}\label{droite-de-marchuxe9-des-capitaux-capital-market-line}}

\begin{itemize}
\item
  A partir des calculs présentés en cours, mettre en oeuvre une méthode
  numérique pour déterminer le portefeuille tangent quand les poids des
  actifs risqués sont contraints à être positifs: \(w_i >= 0\).
\item
  Même calcul en ajoutant des contraintes supplémentaires qui vous
  semblent pertinentes (ex: pas plus de 20\% de l'actif risqué alloué à
  un seul titre, etc.)
\end{itemize}

\end{document}
